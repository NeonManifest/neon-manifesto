\newpage
\section{Magic}
\label{sec:magic}

Characters with the Spell skill are able to use magic. This chapter descibes how magic and spells work in Neon Manifesto.

\subsection{Spell Thematics}
When you create a character or pick up the Spell skill, you choose the distinct general theme or appearence that all your spells will have: what does your magic look like?

\subsection{Learning a Spell}
For every 1 in the Spell Skill value, a character rolls 2d6 in \ref{sec:spell-list} to determine which spell they automatically acquire. Learning or creating new spells can also be done as a project that takes around 20 days and costs three treasures. Characters have the option of choosing their first spell. There are six disciplines of magic in Neon Manifesto:
\begin{itemize}
    \item {\textbf{War}: straightforward combat magic;}
    \item {\textbf{Wizardry}: the domain of wizards, this discipline contains spells that change and manipulate the spiritual properties of objects and characters;}
    \item {\textbf{Illusion}: spells that deceive and elude the senses;}
    \item {\textbf{Divination}: spells around gathering information and knowledge;}
    \item {\textbf{Thaumaturgy}: miraculous spells, this discipline is centered around controlling magic and the soul itself;}
    \item {\textbf{Transmutation}: spells that manipulate the physical qualities of objects and characters.}
\end{itemize}
When characters learn an innate spell, unless they are a Mage, cannot learn spells from opposing disciplines (if they come up in the 2d6 rolls you learn from the opposing discipline they have). The opposing pairs are:
\begin{itemize}
    \item {\textbf{War-Thauaturgy}}
    \item {\textbf{Wizardry-Transmutation}}
    \item {\textbf{Illusion-Divination}}
\end{itemize}

\subsection{Casting a Spell}
In order to cast a spell, make a Casting roll, that is a standard d12 using the Spell skill:
\begin{itemize}
    \item {If the Spell roll comes up with a Natural 1 (by rolling 1 on the die), the spell fizzles, you lose it for the day, and a calamity happens;}
    \item {If the Spell roll fails (by rolling 7 or below), the spell fizzles and you lose it for the day;}
    \item {If the Spell roll succeeds with a degree of I (by rolling 8 to 12), it's I effect is activated, you may choose to lose it for the day or incur 1 WRATH;}
    \item {If the Spell roll succeeds with a degree of II (by rolling 13 to 17), you may activate its I or II effect, you may choose to lose it for the day or incur 2 WRATH;}
    \item {If the Spell roll succeeds with a degree of III (by rolling 18 or more), you may activate its I, II or III effect, you may choose to lose it for the day or incur 4 WRATH;;}
\end{itemize}
Casting spells may accumulate WRATH, which will make everything worse when a calamity happens to the caster. Unlike with standard and attack rolls, you can use BENNIES to subtract what you would add to the roll.

\subsection{Attacking with a Spell}
A creative PLAYER will find that every spell listed in Neon Manifesto can be conceivably used as an attack method. To attack with a spell, make an attack roll using the Spell skill simultaneously as the standard roll using the Spell skill. The attack is determined as normal but the spell roll will give the attack additional consequences:
\begin{itemize}
    \item {If the Casting roll fails, lose the highest-valued half of the dice before any explosions (round up);}
    \item {If the Casting roll succeeds with a degree of I (by rolling 8 to 12), the attack proceeds as normal;}
    \item {If the Casting roll succeeds with a degree of II (by rolling 13 to 17), add a d6 to the attack;}
    \item {If the Casting roll succeeds with a degree of III (by rolling 18 or more), add two d6 to the attack;}
\end{itemize}
The Bolt and Blast spells are special cases in which they are spells best suited for attacking, thus have special rules.

\subsection{Spells by Discipline}
\label{sec:spell-list}
\begin{multicols}{3}
\subsubsection*{1 - War}
\begin{enumerate}
    \item {Bolt}
    \item {Blast}
    \item {Darkness}
    \item {Shield}
    \item {Sword}
    \item {Web}
\end{enumerate}
\subsubsection*{4 - Divination }
\begin{enumerate}
    \item {Portent}
    \item {X-Ray Vision}
    \item {Scrying}
    \item {Olfactory Trace}
    \item {Cartomancy}
    \item {Oneiromancy}
\end{enumerate}
\subsubsection*{2 - Wizardry}
\begin{enumerate}
    \item {Enchantment}
    \item {Magic Circle}
    \item {Summoning}
    \item {Telepathy}
    \item {Ward}
    \item {Astral Projection}
    \item {Telekinesis}
\end{enumerate}
\subsubsection*{5 - Thaumaturgy }
\begin{enumerate}
    \item {Antimagic}
    \item {Emotion}
    \item {Necromancy}
    \item {Command}
    \item {Sealing}
    \item {Miracle}
\end{enumerate}
\subsubsection*{3 - Illusion }
\begin{enumerate}
    \item {Phantasm}
    \item {Mirage}
    \item {Sound}
    \item {Invisibility}
    \item {Mirror Image}
    \item {Masquerade}
\end{enumerate}
\subsubsection*{6 - Transmutation}
\begin{enumerate}
    \item {Polymorph}
    \item {Alchemy}
    \item {Shape Water}
    \item {Shape Earth}
    \item {Gravity}
    \item {Gate}
\end{enumerate}
\end{multicols}
\subsection{Spells in Alphabetical Order}
\begin{multicols}{3}
\subsubsection*{Alchemy}
\subsubsection*{Antimagic}
\subsubsection*{Astral Projection}
\subsubsection*{Blast}
\subsubsection*{Bolt}
\subsubsection*{Cartomancy}
\subsubsection*{Command}
\subsubsection*{Darkness}
\subsubsection*{Emotion}
\subsubsection*{Enchantment}
\subsubsection*{Gate}
\subsubsection*{Gravity}
\subsubsection*{Invisibility}
\subsubsection*{Magic Circle}
\subsubsection*{Masquerade}
\subsubsection*{Mirage}
\subsubsection*{Mirror Image}
\subsubsection*{Miracle}
\subsubsection*{Necromancy}
\subsubsection*{Olfactory Trace}
\subsubsection*{Oneiromancy}
\subsubsection*{Phantasm}
\subsubsection*{Polymorph}
\subsubsection*{Portent}
\subsubsection*{Scrying}
\subsubsection*{Sealing}
\subsubsection*{Shape Earth}
\subsubsection*{Shape Water}
\subsubsection*{Shield}
\subsubsection*{Sound}
\subsubsection*{Summoning}
\subsubsection*{Sword}
\subsubsection*{Telekinesis}
\subsubsection*{Telepathy}
\subsubsection*{Ward}
\subsubsection*{Web}
\subsubsection*{X-Ray Vision}
\end{multicols}

\subsection{Calamities}