\newpage
\section{Foes}
Foes are characters the CONDUCTOR controls that are in opposition to PLAYER-controlled characters. Each FOE may have a statblock, delineating their special abilities. Like PLAYER characters, FOES have a number of HITS they can take before they are defeated and a HIT VALUE that determines how hard they are to take HITS. FOE HITs are defined by their HIT DICE: when a FOE comes into play, roll a number of d6 equal to the HIT DICE value on their statblock, the result is how many HITs they can take before being defeated. If the number of HITs comes up with the maximum value, the FOE is a Champion, and has special rules involving it.

There are 10 FOE types in Neon Manifesto, each representing an archetype of fantasy monsters and other adversaries:
\begin{itemize}
    \item {\textbf{Beasts}: animals -- talking or not -- and animal-like creatures are in this category, known for being physical, natural and brutal threats;}
    \item {\textbf{Cosmic}: things not of this -- or any -- world. Challenge comprehension and the psyche. Psychic creatures and otherworldly horrors populate this type;}
    \item {\textbf{Construct}: imagined foes, created by accident or with a purpose by craftsmanship or happenstance;}
    \item {\textbf{Dragons}: the mightiest of foes, hardly requiring any presentation;}
    \item {\textbf{Faerie}: creatures of rich culture and great whimsy, always scheming to get on top;}
    \item {\textbf{Humanoids}: civilized foes that are not beasts or imagined, men at arms, bandits and the like;}
    \item {\textbf{Plants}: plant life capable of acting and fighting;}
    \item {\textbf{Shapeshifters}: creatures without a fixed form and the ability to transform;}
    \item {\textbf{Spirits}: godlings, angels, demons, genies, all sorts of spiritual beings are in this category;}
    \item {\textbf{Undead}: creatures with characteristics of living and dead, hungering for the heat of life.}
\end{itemize}

\subsection{Champion Foes}
If the number of HITs on a FOE's HIT DICE roll comes up with the maximum value (e.g. a foe with 2 HIT DICE has 12 HITS), the FOE is a Champion. Every champion has a name and a feat: a fact in its personal history that made it particularly infamous and/or powerful. Feats may grant the FOE special abilities, and can be determined by rolling on the feat table by monster type.

Foes with one HIT DIE cannot be made champions from their roll.

\subsection{Alphabetical Index of Foes}
\begin{multicols}{3}
    \begin{itemize}
        \item Alraune \pageref{foe:alraune}
        \item Amorph \pageref{foe:amorph}
        \item Angel \pageref{foe:angel}
        \item Assassin \pageref{foe:assassin}
        \item Asura \pageref{foe:asura}
        \item Barrow Wight \pageref{foe:barrow-wight}
        \item Bear \pageref{foe:bear}
        \item Behemoth \pageref{foe:behemoth}
        \item Blaster \pageref{foe:blaster}
        \item Bomb \pageref{foe:bomb}
        \item Bullet \pageref{foe:bullet}
        \item Cephalocordate \pageref{foe:cephalocordate}
        \item Combatant \pageref{foe:combatant}
        \item Doll \pageref{foe:doll}
        \item Dragon, Fire \pageref{foe:dragon-fire}
        \item Dragon, Frost \pageref{foe:dragon-frost}
        \item Dragon, Poison \pageref{foe:dragon-poison}
        \item Dragon, Shadow \pageref{foe:dragon-shadow}
        \item Dragon, Storm \pageref{foe:dragon-storm}
        \item Eater of Minds \pageref{foe:eater-of-minds}
        \item Elk \pageref{foe:elk}
        \item Elf \pageref{foe:elf}
        \item False Idol \pageref{foe:false-idol}
        \item Forlorn \pageref{foe:forlorn}
        \item Gazer \pageref{foe:gazer}
        \item Giant Eagle \pageref{foe:giant-eagle}
        \item Giant Spider \pageref{foe:giant-spider}
        \item Goblin \pageref{foe:goblin}
        \item Golem \pageref{foe:golem}
        \item Greater Demon \pageref{foe:greater-demon}
        \item Imperceptible \pageref{foe:imperceptible}
        \item Knight \pageref{foe:knight}
        \item Kobold \pageref{foe:kobold}
        \item Lesser Demon \pageref{foe:lesser-demon}
        \item Master Brain \pageref{foe:master-brain}
        \item Mimic \pageref{foe:mimic}
        \item Naga \pageref{foe:naga}
        \item Poltergeist \pageref{foe:poltergeist}
        \item Rabble \pageref{foe:rabble}
        \item Rakshasa \pageref{foe:rakshasa}
        \item Ranger \pageref{foe:ranger}
        \item Salamander \pageref{foe:salamander}
        \item Shambling Shrubbery \pageref{foe:shambling}
        \item Shark \pageref{foe:shark}
        \item Sidhe \pageref{foe:sidhe}
        \item Skeleton \pageref{foe:skeleton}
        \item Sylph \pageref{foe:sylph}
        \item Therianthrope \pageref{foe:therianthrope}
        \item Titan \pageref{foe:titan}
        \item Treant \pageref{foe:treant}
        \item Undead Mage \pageref{foe:undead-mage}
        \item Undine \pageref{foe:undine}
        \item War Machine \pageref{foe:war-machine}
        \item Wizard \pageref{foe:wizard}
        \item Wolf \pageref{foe:wolf}
        \item Wraith \pageref{foe:wraith}
        \item Zombie \pageref{foe:zombie}
    \end{itemize}
\end{multicols}

\subsection{Beasts}
\begin{multicols}{2}
    \subsubsection*{Bear}\label{foe:bear}
    The archetypical monster. There are at least three variations of bear: one of them too small; one of them too large; and the other just right.
    \begin{tabularx}{\columnwidth}{|X|X|X|X|}
        \hline
        & Black & Brown & Cave \\
        \hline
        HIT VALUE & 3 & 4 & 4 \\
        \hline
        HIT DICE & 2 & 4 & 6 \\
        \hline
        SKILLS & Claw 2, Climbing, Swimming and Running 3 & Claw 3, Bite 2, Climbing, Swimming and Running 4 & Claw 4, Bite 3, Climbing, Swimming and Running 5 \\
        \hline
        SPECIAL & Panic & Panic & Panic, Rend \\
        \hline
    \end{tabularx}

    \textbf{Panic:} When a Bear is hit the first time in a fight, it will panic for the rest of it. A Panicking bear can make two attacks using its claws and one using its bite every round.
    
    \textbf{Rend:} If a Cave Bear hits the same target with its three attack rolls in the same round, all hits caused by the Cave Bear on the round gain the \textbf{hemorrhage} condition.
    \subsubsection*{Behemoth}\label{foe:behemoth}
    \subsubsection*{Elk}\label{foe:elk}
    \subsubsection*{Giant Eagle}\label{foe:giant-eagle}
    \subsubsection*{Giant Spider}\label{foe:giant-spider}
    \subsubsection*{Shark}\label{foe:shark}
    \subsubsection*{Wolf}\label{foe:wolf}
\end{multicols}
\subsubsection{Champion Beasts}

\subsection{Cosmic}
\begin{multicols}{2}
    \subsubsection*{Cephalocordate}\label{foe:cephalocordate}
    \subsubsection*{Eater of Minds}\label{foe:eater-of-minds}
    \subsubsection*{False Idol}\label{foe:false-idol}
    \subsubsection*{Forlorn}\label{foe:forlorn}
    \subsubsection*{Gazer}\label{foe:gazer}
    \subsubsection*{Imperceptible}\label{foe:imperceptible}
    \subsubsection*{Master Brain}\label{foe:master-brain}
\end{multicols}
\subsubsection{Champion Cosmic}

\subsection{Construct}
\begin{multicols}{2}
    \subsubsection*{Bomb}\label{foe:bomb}
    \subsubsection*{Blaster}\label{foe:blaster}
    \subsubsection*{Bullet}\label{foe:bullet}
    \subsubsection*{Doll}\label{foe:doll}
    \subsubsection*{Golem}\label{foe:golem}
    \subsubsection*{Titan}\label{foe:titan}
    \subsubsection*{War Machine}\label{foe:war-machine}
\end{multicols}
\subsubsection{Champion Constructs}

\subsection{Dragons}
\begin{multicols}{2}
    \subsubsection*{Dragon, Fire}\label{foe:dragon-fire}
    \subsubsection*{Dragon, Frost}\label{foe:dragon-frost}
    \subsubsection*{Dragon, Poison}\label{foe:dragon-poison}
    \subsubsection*{Dragon, Shadow}\label{foe:dragon-shadow}
    \subsubsection*{Dragon, Storm}\label{foe:dragon-storm}
    \subsubsection*{Kobold}\label{foe:kobold}
\end{multicols}
\subsubsection{Champion Dragons}

\subsection{Faerie}
\begin{multicols}{2}
    \subsubsection*{Elf}\label{foe:elf}
    \subsubsection*{Goblin}\label{foe:goblin}
    \subsubsection*{Salamander}\label{foe:salamander}
    \subsubsection*{Sidhe}\label{foe:sidhe}
    \subsubsection*{Sylph}\label{foe:sylph}
    \subsubsection*{Undine}\label{foe:undine}
\end{multicols}
\subsubsection{Champion Faerie}

\subsection{Humanoids}
\begin{multicols}{2}
    \subsubsection*{Assassin}\label{foe:assassin}
    \subsubsection*{Combatant}\label{foe:combatant}
    \subsubsection*{Knight}\label{foe:knight}
    \subsubsection*{Rabble}\label{foe:rabble}
    \subsubsection*{Ranger}\label{foe:ranger}
    \subsubsection*{Wizard}\label{foe:wizard}
\end{multicols}
\subsubsection{Champion Humanoids}

\subsection{Plants}
\begin{multicols}{2}
    \subsubsection*{Alraune}\label{foe:alraune}
    \subsubsection*{Shambling Shrubbery}\label{foe:shambling}
    \subsubsection*{Treant}\label{foe:treant}
\end{multicols}
\subsubsection{Champion Plants}

\subsection{Shapeshifters}
\begin{multicols}{2}
    \subsubsection*{Amorph}\label{foe:amorph}
    \subsubsection*{Mimic}\label{foe:mimic}
    \subsubsection*{Therianthrope}\label{foe:therianthrope}
\end{multicols}
\subsubsection{Champion Shapeshifters}

\subsection{Spirits}
\begin{multicols}{2}
    \subsubsection*{Angel}\label{foe:angel}
    \subsubsection*{Asura}\label{foe:asura}
    \subsubsection*{Greater Demon}\label{foe:greater-demon}
    \subsubsection*{Lesser Demon}\label{foe:lesser-demon}
    \subsubsection*{Naga}\label{foe:naga}
    \subsubsection*{Rakshasa}\label{foe:rakshasa}
\end{multicols}
\subsubsection{Champion Spirits}

\subsection{Undead}
\begin{multicols}{2}
    \subsubsection*{Barrow Wight}\label{foe:barrow-wight}
    \subsubsection*{Poltergeist}\label{foe:poltergeist}
    \subsubsection*{Skeleton}\label{foe:skeleton}
    \subsubsection*{Lich}\label{foe:undead-mage}
    \subsubsection*{Wraith}\label{foe:wraith}
    \subsubsection*{Zombie}\label{foe:zombie}
\end{multicols}
\subsubsection{Champion Undead}