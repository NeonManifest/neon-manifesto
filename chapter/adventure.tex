\section{Rules and Procedures of Adventure}

\subsection{Standard Roll Procedure}
\begin{enumerate}
    \item {The PLAYER describes what their character is doing, if the CONDUCTOR identifies that action as having a chance of failure and either success and failure would change the character's situation in the narrative, they may call for a roll;}
    \item {The CONDUCTOR asks the PLAYER whether they have any skills that could help in this situation;}
    \item {The CONDUCTOR sets the difficulty for the roll;}
    \item {The CONDUCTOR sets the stakes for a roll, and the PLAYER is given an opportunity to raise them;}
    \item {The player rolls a d12 and adds their skill, attempting to roll over the difficulty of the roll;}
    \item {After seeing the roll's result, the PLAYER may supplement its result using BENNIES, each BENNY spent grants a +1 to the roll}
    \item {The CONDUCTOR describes the outcome of the roll.}
\end{enumerate}

\subsection{Attack Rolls}
Attack rolls are a special kind of roll where the action being described consists of a character attempting to harm another, physically or otherwise. The procedure for attack rolls differs slightly from standard rolls.
\begin{enumerate}
    \item {The PLAYER initates saying something like ``I attack'' or ``My character attacks'' or by describing an action that qualifies as an attack. The attack's target must be in condition of defending themselves and fighting, otherwise the roll is not an attack;}
    \item {The CONDUCTOR asks the PLAYER whether they have any skills that could help in their chosen attack method;}
    \item {The roll's difficulty is set by the FOE being attacked;}
    \item {The player rolls a 1d6 + Xd6 where X is the relevant skill's value, attempting to roll over the FOE's difficulty;}
    \item {After seeing the roll's result, the PLAYER may supplement its result using BENNIES, each BENNY spent grants a +1 to the roll}
    \item {The CONDUCTOR describes the outcome of the roll using the following guidelines:}
    \begin{itemize}
        \item {\textbf{Each die equal to or above the FOE's difficulty} causes 1 hit;}
        \item {\textbf{Each die resulting in 6 explodes!} Causing 1 hit and adding another d6 to the roll;}
        \item {\textbf{If no d6 causes a hit, the FOE will immediately return with an attack of its own}.}
    \end{itemize}
\end{enumerate}
Foe attacks work just like player attacks.

\subsection{Skill Descriptions}
\subsubsection*{Acrobatics}
Used for rolling, balancing, falling in style, jumping, chandelier swinging, etc.
\subsubsection*{Awareness (S)}
Use this to perceive things normally hidden or out of ordinary, to spot traps, and to use all of the character's senses to the best of their ability. A character can specify a sense through which their awareness works, specifying gives a +1 to the skill but makes them unable to use it outside its specification (unless they have the general skill).
\subsubsection*{Art/Craft (S)}
A skill that describes doing a job: be it blacksmithing; painting; fishing; any job that is not covered by the standard skill set. In order to take this skill, a character \textbf{must} specity which art or craft they are good at, specifying gives a +1 to the skill.
\subsubsection*{Calculus}

\subsubsection*{Climbing, Swimming and Running}
\subsubsection*{Disguise}
\subsubsection*{Evaluate}
\subsubsection*{Healing}
\subsubsection*{Languages}
\subsubsection*{Melee Weapon}
\subsubsection*{Missile Weapon}
\subsubsection*{Locks and Traps}
\subsubsection*{Poison}
\subsubsection*{Rhetoric}   
\subsubsection*{Script}
\subsubsection*{Skulduggery}
\subsubsection*{Spell}
\subsubsection*{Strength}
\subsubsection*{Throwing Weapon}
\subsubsection*{Other skills!}