\section{Rules and Procedures of Adventure}

\subsection{Standard Roll Procedure}
\begin{enumerate}
    \item {The PLAYER describes what their character is doing, if the CONDUCTOR identifies that action as having a chance of failure and either success and failure would change the character's situation in the narrative, they may call for a roll;}
    \item {The CONDUCTOR asks the PLAYER whether they have any skills that could help in this situation;}
    \item {The CONDUCTOR sets the difficulty for the roll;}
    \item {The player rolls a d12 and adds their skill, attempting to roll over the difficulty of the roll;}
    \item {After seeing the roll's result, the PLAYER may supplement its result using BENNIES, each BENNY spent grants a +1 to the roll}
    \item {The CONDUCTOR describes the outcome of the roll.}
\end{enumerate}

\subsection{Wilderness Exploration Procedure}
The \textbf{unit of time} during wilderness exploration is called a \textbf{phase}. There are three \textbf{phases} in a day:
\begin{itemize}
    \item {\textbf{Morning}: From dawn to noon;}
    \item {\textbf{Afternoon}: From noon to dusk;}
    \item {\textbf{Night}: From dusk to dawn;}
\end{itemize}
During each phase, each character can take one action:
\begin{itemize}
    \item {\textbf{Standard}: each character may take actions individually, wilderness actions can be anything that makes sense in the shared narrative, some sample actions are:
    \begin{itemize}
        \item {\textbf{Forage}: a character makes a difficulty 7 roll using their survival skill. If succeeding the supply check for the day is not made, and the party will not take a hit due to their supplies being depleted;}
        \item {\textbf{Navigation}: a character discovers nearby hexes, spotting mountains up to 4 hexes away, civilization up to 2 hexes away, and the terrain types of neighboring hexes;}
        \item {\textbf{Scouting}: a character searches the hex, discovering its adventure and possibly an advantageous angle for approaching it;}
        \item {\textbf{Sentinel}: a character in sentinel keeps watch over the party to ensure safety for the phase. The CONDUCTOR may prompt the character to make a roll using the Awareness skill to perceive danger proactively;}
        \item {\textbf{Rest}: a character in wilderness exploration that doesn't take the rest action at least once a day will take a hit that will not recover until rest is taken.}
    \end{itemize}
    }
    \item {\textbf{Travel}: the whole party decides on a travel action to perform, and move around the world accordingly. The unit of distance in wilderness traveling is a \textbf{hex}, the main travel actions are:
    \begin{itemize}
        \item {\textbf{Trek}: The party travels 1 hex;}
        \item {\textbf{Gallop}: can be used if the whole party is mounted. The party travels 2 hexes, mounts will get tired unless you're entering or leaving civilization this phase. Tired mounts cannot gallop;}
        \item {\textbf{Ferry}: can be used on a water vehicle. The party travels 3 hexes.}
    \end{itemize}
    }
\end{itemize}
When ending a phase in wilderness, make a wilderness roll (d6):
\begin{itemize}
    \item {\textbf{1 - Encounter an adventure};}
    \item {\textbf{2-3 - Encounter clues for the hex's adventure and landmark};}
    \item {\textbf{4-6 - Encounter the hex's landmark};}
\end{itemize}
At the end of each day in the wilderness, the travelling party rolls to see if their supplied. Supply starts as a d12 roll, and the party's leader must roll above 3 to keep supplies from dwidling. Failing this roll will decrease the supply die to a d10, and each failure reduces the dice size (d12 $\rightarrow$ d10 $\rightarrow$ d8 $\rightarrow$ d6 $\rightarrow$ d4). Once a supply roll fails at d4, the party's supplies have been depleted. Each member of a party whose supplies have been depleted take a hit that will not recover until the party is fed again.

\subsection{Dungeon Exploration Procedure}
The \textbf{unit of time} during dungeon exploration is called a \textbf{turn}. The amount of time a \textbf{turn} takes is highly variable and subjective. Each character can perform an action each turn. The turn procedures as follows:
\begin{enumerate}
    \item {The conductor ticks light. At every six ticks, the exploring party rolls to maintain their light. Light starts as a d12 roll, and the party's leader must roll above 3 to keep the light from dwidling. Failing this roll will decrease the light die to a d10, and each failure reduces the dice size (d12 $\rightarrow$ d10 $\rightarrow$ d8 $\rightarrow$ d6 $\rightarrow$ d4). Once a light roll fails at d4, the party's light has been totally depleted and the party is in the dark. Things get FREAKY in the dark.}
    \item {The condcutor describes the location the characters are;}
    \item {Each character takes an action, which can be anything that makes sense in the narrative, sample actions are:
    \begin{itemize}
        \item {\textbf{Move to another room or hallway};}
        \item {\textbf{Search a 10-foot (6 meter) square};}
        \item {\textbf{Interact with something the conductor describes};}
    \end{itemize}
    }
\end{enumerate}

\subsection{Combat Procedure}

\subsubsection{Attack Rolls}
Attack rolls are a special kind of roll where the action being described consists of a character attempting to harm another, physically or otherwise. The procedure for attack rolls differs slightly from standard rolls.
\begin{enumerate}
    \item {The PLAYER initates saying something like ``I attack'' or ``My character attacks'' or by describing an action that qualifies as an attack. The attack's target must be in condition of defending themselves and fighting, otherwise the roll is not an attack;}
    \item {The CONDUCTOR asks the PLAYER whether they have any skills that could help in their chosen attack method;}
    \item {The roll's difficulty is set by the FOE being attacked;}
    \item {The player rolls a 1d6 + Xd6 where X is the relevant skill's value, attempting to roll over the FOE's difficulty;}
    \item {After seeing the roll's result, the PLAYER may supplement its result using BENNIES, each BENNY spent grants a +1 to the roll of one die;}
    \item {The CONDUCTOR describes the outcome of the roll using the following guidelines:}
    \begin{itemize}
        \item {\textbf{Each die equal to or above the FOE's difficulty} causes 1 hit;}
        \item {\textbf{Each die resulting in 6 or above explodes!} Causing 1 hit and adding another d6 to the roll;}
        \item {\textbf{If no d6 causes a hit, the FOE will immediately return with an attack of its own}.}
    \end{itemize}
\end{enumerate}
Foe attacks work just like player attacks.

\subsection{When things get FREAKY}
\begin{itemize}
    \item {Fights are resolved as one-roll fights.}
    \item {The difficulty of every skill check is 12.}
    \item {Moving to another room or hallway is the only dungeon action allowed.}
    \item {Supply checks in the wilderness instantly fail.}
    \item {Trek, Gallop and Ferry are the only wilderness actions allowed.}
\end{itemize}

\subsubsection{One-roll fights}

\subsection{Skill Descriptions}
\begin{multicols}{2}
\subsubsection*{Acrobatics}
Used for rolling, balancing, falling in style, jumping, chandelier swinging, etc.
\subsubsection*{Awareness (S)}
Use this to perceive things normally hidden or out of ordinary, to spot traps, and to use all of the character's senses to the best of their ability. A character can specify a sense through which their awareness works, specifying gives a +1 to the skill but makes them unable to use it outside its specification (unless they have the general skill). This skill can be used in attacks that involve aiming carefully.
\subsubsection*{Art/Craft (S)}
A skill that describes doing a job: be it blacksmithing; painting; fishing; any job that is not covered by the standard skill set. In order to take this skill, a character \textbf{must} specity which art or craft they are good at, specifying gives a +1 to the skill.
\subsubsection*{Calculus}
Represents the ability to detect fulcrum points, architectural stability, predict the motions of bodies, all of the math and physics that takes extraordinary skill to do on the fly.
\subsubsection*{Climbing, Swimming and Running}
Represents the ability to move quickly and safely through terrain and water.
\subsubsection*{Disguise}
The ability to impersonate someone else or play a role.
\subsubsection*{Evaluate}
Describes the ability to ascertain the value of an object and its true characteristics, mundane, magic, spiritual, etc.
\subsubsection*{Healing}
Use this to undo harm to a character. To heal a character, make an attack roll against the foe that caused the harm's difficulty (or 3 if not specified) each hit recovers a hit. Rolling under the difficulty with all the dice causes 1 hit of damage to the character being healed. Harm from different sources must be treated separately. Healing a character takes a unit of \textbf{time}.
\subsubsection*{Languages}
Represents knowledge in the many tongues of the world. Use this for comprehending and speaking foreign, liturgic or ancient languages.
\subsubsection*{Manuality}
Use this to pick pockets, sleight of hand, perform in stage magic, and make attacks using skill and grace.
\subsubsection*{Mechanisms and Devices}
This skill measures the ability to interact with locks, traps, and the like.
\subsubsection*{Poison}
A skill that abstracts the making and using of poison. When creating a posion, make an attack roll against difficulty 3, the number of hits is equal to the difficulty of the foe the poison is effective for. Poison creation takes a unit of \textbf{time} and a unit of \textbf{loot}. Only make an attack using poison if you are actively using it in a fight (e.g. through a poisoned weapon).
\subsubsection*{Rhetoric}
Use this to express yourself in spoken and nonverbal language. To present arguments, and to strike with words in situations where its possible.
\subsubsection*{Script}
Use this to express yourself in written language, includes the making of both legitimate and forged documents. If you are creative, this can be used as an attack method.
\subsubsection*{Spell}
The abiltiy to perform magic spells, can also serve as an attack method when using an attack spell. Chapter \ref{sec:magic} explains the usage of this skill in detail.
\subsubsection*{Strength}
Use this to lift heavy objects, bend bars, break doors and the like. This skill is also a reliable attack method.
\subsubsection*{Survival}
Use this for foraging while travelling in wilderness.
\subsubsection*{Weapon (S)}
Use this to fight with a weapon. When picking this skill, the character \textbf{must} specify which weapon they are good at. Specifying gives the character +1 to the skill.
\subsubsection*{Other skills}
There are as many skills as there are abilities that might be relevant to characters in a game setting. Players should feel free to discuss adding them to characters if they make sense in the shared fiction.
\end{multicols}

\subsection{Ending a Session}

\begin{itemize}
    \item {Ten minutes before the end of the session, things get FREAKY;}
\end{itemize}