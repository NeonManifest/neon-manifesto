\section{Rules and Procedures of Adventure}
\label{sec:adventure}
The standard play procedure is as follows:
\begin{itemize}
    \item The CONDUCTOR describes the situation, setting the stage for the players;
    \item A PLAYER declares an action, describing what their character does to respond to or interact with the situation;
    \item Die rolls are made as necessary to determine the outcome of the declared action;
    \item The CONDUCTOR narrates the result of the action, updating the situation based on the outcome;
    \item The play procedure begins anew.
\end{itemize}

\subsection{Standard Roll Procedure}
When the outcome of an action is uncertain, the roll procedure is called as follows:
\begin{enumerate}
    \item {The PLAYER describes what their character is doing, if the CONDUCTOR identifies that action as having a chance of failure and either success and failure would change the character's situation in the narrative, they may call for a roll;}
    \item {The CONDUCTOR asks the PLAYER whether they have any skills that could help in this situation;}
    \item {The CONDUCTOR sets the difficulty for the roll;}
    \item {The player rolls a d12 and adds their skill, attempting to roll over the difficulty of the roll;}
    \item {After seeing the roll's result, the PLAYER may supplement its result using BENNIES, each BENNY spent grants a +1 to the roll}
    \item {The CONDUCTOR describes the outcome of the roll.}
    \item {If the player used a skill, they take note of it.}
\end{enumerate}

\subsection{Wilderness Exploration Procedure}
When the PLAYER characters explore the wilderness, this procedure activates. The \textbf{unit of time} during wilderness exploration is called a \textbf{phase}. There are three \textbf{phases} in a day:
\begin{itemize}
    \item {\textbf{Morning}: From dawn to noon;}
    \item {\textbf{Afternoon}: From noon to dusk;}
    \item {\textbf{Night}: From dusk to dawn;}
\end{itemize}
During each phase, each character can take one action:
\begin{itemize}
    \item {\textbf{Standard}: each character may take actions individually, wilderness actions can be anything that makes sense in the shared narrative, some sample actions are:
    \begin{itemize}
        \item {\textbf{Forage}: a character makes a difficulty 7 roll using their survival skill. If succeeding the supply check for the day is not made, and the party will not take a hit due to their supplies being depleted;}
        \item {\textbf{Navigation}: a character discovers nearby hexes, spotting mountains up to 4 hexes away, civilization up to 2 hexes away, and the terrain types of neighboring hexes;}
        \item {\textbf{Scouting}: a character searches the hex, discovering its adventure and possibly an advantageous angle for approaching it;}
        \item {\textbf{Sentinel}: a character in sentinel keeps watch over the party to ensure safety for the phase. The CONDUCTOR may prompt the character to make a roll using the Awareness skill to perceive danger proactively;}
        \item {\textbf{Rest}: a character in wilderness exploration that doesn't take the rest action at least once a day will take a hit that will not recover until rest is taken.}
    \end{itemize}
    }
    \item {\textbf{Travel}: the whole party decides on a travel action to perform, and move around the world accordingly. The unit of distance in wilderness traveling is a \textbf{hex}, the main travel actions are:
    \begin{itemize}
        \item {\textbf{Trek}: The party travels 1 hex;}
        \item {\textbf{Gallop}: can be used if the whole party is mounted. The party travels 2 hexes, mounts will get tired unless you're entering or leaving civilization this phase. Tired mounts cannot gallop;}
        \item {\textbf{Ferry}: can be used on a water vehicle. The party travels 3 hexes.}
    \end{itemize}
    }
\end{itemize}
When ending a phase in wilderness, make a wilderness roll (d6):
\begin{itemize}
    \item {\textbf{1 - Encounter an adventure};}
    \item {\textbf{2-3 - Encounter clues for the hex's adventure and landmark};}
    \item {\textbf{4-6 - Encounter the hex's landmark};}
\end{itemize}
At the end of each day in the wilderness, the travelling party rolls to see if their supplied. Supply starts as a d12 roll, and the party's leader must roll above 3 to keep supplies from dwidling. Failing this roll will decrease the supply die to a d10, and each failure reduces the dice size (d12 $\rightarrow$ d10 $\rightarrow$ d8 $\rightarrow$ d6 $\rightarrow$ d4). Once a supply roll fails at d4, the party's supplies have been depleted. Each member of a party whose supplies have been depleted take a hit that will not recover until the party is fed again.

\subsection{Dungeon Exploration Procedure}
When the PLAYER characters explore the underworld, this procedure activates. The \textbf{unit of time} during dungeon exploration is called a \textbf{turn}. The amount of time a \textbf{turn} takes is highly variable and subjective. Each character can perform an action each turn. The turn procedures as follows:
\begin{enumerate}
    \item {The conductor ticks light. At every six ticks, the exploring party rolls to maintain their light. Light starts as a d12 roll, and the party's leader must roll above 3 to keep the light from dwidling. Failing this roll will decrease the light die to a d10, and each failure reduces the dice size (d12 $\rightarrow$ d10 $\rightarrow$ d8 $\rightarrow$ d6 $\rightarrow$ d4). Once a light roll fails at d4, the party's light has been totally depleted and the party is in the dark. Things get FREAKY in the dark.}
    \item {The condcutor describes the location the characters are;}
    \item {Each character takes an action, which can be anything that makes sense in the narrative, sample actions are:}
    \begin{itemize}
        \item {\textbf{Move to another room or hallway};}
        \item {\textbf{Search a 10-foot (6 meter) square};}
        \item {\textbf{Interact with something the conductor describes};}
        \item {\textbf{Rest};}
    \end{itemize}
\end{enumerate}

\subsection{Haven Procedure}
A haven is any safe patch of civilization where characters can take restful activity, a haven turn happens when PLAYER characters reach one and decide to spend time there. The \textbf{unit of time} in haven is days, during which characters can take one action. Sample actions are as follows:
\begin{itemize}
    \item {\textbf{Prepare}: Each party member gains a BENNY (max 3 per character);}
    \item {\textbf{Work on a project}: a character rolls using a relevant skill for the project. If successful, they mark a progress in the project. Finishing a project requires taking this action after progress has been filled;}
    \item {\textbf{Gather information}: a character discovers two rumors about the region: one true and one false. They do not know which one is which;}
    \item {\textbf{Carouse}: trade a treasure for debauchery and sin. Gain 1 EXPERIENCE. Roll a d12, if the result is above the carousing character's level, a mishap happens. Appendix \ref{ap:carousing} cointains carousing mishaps tables for each class;}
    \item {\textbf{Gear up}: the character acquires up to 3 pieces of adventuring gear;}
\end{itemize}
Characters recover 1 HIT and gain 1 BENNY every haven turn.

\subsection{Projects}
Each project has an objective (i.e. what the project is trying to reach) and a time (measured in haven days). Projects are discussed between PLAYER and CONDUCTOR but can be
\begin{itemize}
    \item Create or find a specific treasure, like a magic item;
    \item Brew a potion;
    \item Create a new spell;
    \item Acquire advanced adventuring gear, such as an airship or mounts;
    \item Help or sabotage a faction;
\end{itemize}

\subsection{Demesne Procedure}
A demesne is a territory controlled and maintained by a character, a demesne turn happens when CHARACTERS reach a demesne and decide to spend time there. Typically, parties of level 4 and above might acquire a demesne as part of the story. The \textbf{unit of time} in demesne action is seasons, during which characters can take two actions. Sample actions are as follows:
\begin{itemize}
    \item {\textbf{Work on a project}: any project that had its time measured in days can be completed in a demesne action;}
    \item {\textbf{Work on a building}: add a new installation to your demesne;}
    \item {\textbf{Hire a specialist}: hire a specialist to manage a building;}
    \item {\textbf{Hire a warband}: hire 12 ordinary soldiers, they can act as a unit;}
    \item {\textbf{Command an operation}: move warbands you control to perform a military operation;}
    \item {\textbf{Establish a successor}: create a new character, this character will stay in the demesne and inherit half of the acting character's EXPERIENCE when the acting character dies;}
    \item {\textbf{Research}: discover about a new faction, realm, or region;}
\end{itemize}
Characters recover all hits every demesne turn and gain MAX BENNIES. After taking the final demesne turn before transitioning to other procedure of play, a haven turn takes place. Factions may advance their objectives.

\subsubsection{Factions}
Factions are powerful forces that are managed by the CONDUCTOR. There are around three factions acting in each realm. Factions have goals they must accomplish and resources to accomplish them. Each faction has a statblock containing:
\begin{itemize}
    \item {\textbf{The faction's roster}: an encounter table of faction's rank and file and elite members;}
    \item {\textbf{The faction's territory}: hexes the faction occupies of competes for;}
    \item {\textbf{The faction's objectives};}
    \item {\textbf{The faction's resources};}
    \item {\textbf{The faction's friends and foes, including the player party};}
\end{itemize}
Each goal has a progress requirement associated with it. At the end of every season, each faction rolls 1d6 for each objective, adding +1 for every resource they have that is relevant to the objective. On a 4 or more, mark progress on the objective. When an objective is completed, it becomes a resource. Factions with three or more resources can field warbands.

\subsection{Combat Procedure}
When one side wants to impose power over another, resisting, side, combat ensues. Combat may be perpetrated through fighting but also through debate or other situations where a party aims to overpower the other. The unit of time in combat is a largely subjective measurement of time called \textbf{rounds}. Combat flows just like the standard play procedure: with the CONDUCTOR describing the situation; A PLAYER acts describing what their characters do to react to it; roll procedures happening; and then the play loop begins anew.

In combat, it's important that each PLAYER acts at least once each round. Players act in a clockwise order, starting with a player chosen by the CONDUCTOR or determined by the situation. A round elapses when every PLAYER-controlled character that can act has acted and every FOE that can act has acted.

In combat, it's common for characters and foes to activate Attack Rolls in order to harm each other. Combat ends when one of the parties is unable, or unwilling, to participate in it. The party that does not fold from combat in the end WINS and gains power over the losing party.

Gaining power over the losing party primarily grants narrative leverage. This could include slaying the opposing party, driving them off, or other story-driven outcomes. While the victors may impose terms or secure control in various ways, the game explicitly avoids themes of slavery, sexual violence, or any equivalent dynamics. Players and the CONDUCTOR are encouraged to frame outcomes with creativity and sensitivity to the game's tone and themes.

\subsubsection{Attack Rolls}
Attack rolls are a special kind of roll where the action being described consists of a character attempting to harm another, physically or otherwise. The procedure for attack rolls differs slightly from standard rolls.
\begin{enumerate}
    \item {The PLAYER initates saying something like ``I attack'' or ``My character attacks'' or by describing an action that qualifies as an attack. The attack's target must be in condition of defending themselves and fighting, otherwise the roll is not an attack;}
    \item {The CONDUCTOR asks the PLAYER whether they have any skills that could help in their chosen attack method;}
    \item {The roll's difficulty is set by the FOE being attacked's HIT VALUE (which is presented in its statblock, or in PLAYER character's cases the character sheet;}
    \item {The player rolls a Xd6 (mininum 1) where X is the relevant skill's value, attempting to roll over the FOE's difficulty;}
    \item {If the situation is advantageous to the attacker, they add another 1d6 to the roll. If it's disadvantageous, they remove 1d6 (the attack automatically fails if its roll is 0d6);}
    \item {After seeing the roll's result, the PLAYER may supplement its result using BENNIES, each BENNY spent grants a +1 to the roll of one die;}
    \item {The CONDUCTOR describes the outcome of the roll using the following guidelines:}
    \begin{itemize}
        \item {\textbf{Each die equal to or above the FOE's HIT VALUE} causes 1 hit;}
        \item {\textbf{Each die resulting in 6 or above explodes!} Causing 1 hit and adding another d6 to the roll;}
        \item {\textbf{If no d6 causes a hit, the FOE will immediately return with an attack of its own}, the PLAYER gains a BENNY.}
    \end{itemize}
\end{enumerate}
Foe attacks work just like player attacks.

\subsubsection{Conditions}
Some foes, spells, and adventuring gear are able to enhance their attacks with conditions. A condition is associated with one or more HITS causing it, and those cannot be recovered unless the condition is cleared first. Conditions feature the following characteristics:
\begin{itemize}
    \item {\textbf{Clearing Condition}: actions that have to be taken within the narrative for the HIT to be recoverable;}
    \item {\textbf{Condition Effect}: a continuous effect that affects the character .}
\end{itemize}

Appendix \ref{app:conditions} presents conditions used by the foes, spells, and adventuring gear.

\subsection{When things get FREAKY}
\begin{itemize}
    \item {Fights are resolved as one-roll fights.}
    \item {The difficulty of every standard d120 check -- including rolls using skills -- is 12.}
    \item {Moving to another room or hallway is the only dungeon action allowed.}
    \item {Supply checks in the wilderness always fail.}
    \item {Trek, Gallop and Ferry are the only wilderness actions allowed;}
    \item {If in the underworld or wilderness, conditions cannot be cleared.}
\end{itemize}

\subsubsection{One-roll fights}
\begin{enumerate}
    \item {Each side rolls a d6, adding:}
    \begin{itemize}
        \item {+1 for each combatant with 4 or more HIT VALUE;}
        \item {+1 if the side outnumbers the other; +2 if the side outnumber the other 2-to-1, +3 if the side outnumber the other, etc;}
        \item {+1 for each magic user who casts a combat-relevant spell (such as a fireball) during the fight;}
        \item {+1 for each of the following advantages each side has. Such as cover, high ground, formation that can't be flanked, surprise, superior firepower, poison, traps, etc.}
    \end{itemize}
    \item {The side that rolls highest wins. Ties are rolled again.}
    \item {Each PLAYER character (regardless of winning or not) rolls a d6, if they roll over their HIT VALUE, they take a number of hits equal to the number rolled;}
    \item {Each PLAYER character on the losing side rolls a d6, if they roll over their HIT VALUE, they are slain or captured;}
    \item {Each CONDUCTOR character on the losing side is slain or driven off.}
\end{enumerate}

\subsection{Skill Descriptions}
Each character begins with a set of skills, determined by their class (for PLAYER characters) or their statblock (CONDUCTOR characters);
\begin{multicols}{2}
\subsubsection*{Acrobatics}
Used for rolling, balancing, falling in style, jumping, chandelier swinging, etc.
\subsubsection*{Awareness}
Use this to perceive things normally hidden or out of ordinary, to spot traps, and to use all of the character's senses to the best of their ability. This skill can be used in attacks that involve aiming carefully.
\subsubsection*{Art/Craft (S)}
A skill that describes doing a job: be it blacksmithing; painting; fishing; any job that is not covered by the standard skill set. In order to take this skill, a character \textbf{must} specify which art or craft they are good at, specifying gives a +1 to standard d12 rolls using this skill.
\subsubsection*{Calculus}
Represents the ability to detect fulcrum points, architectural stability, predict the motions of bodies, all of the math and physics that takes extraordinary skill to do on the fly.
\subsubsection*{Climbing, Swimming and Running}
Represents the ability to move quickly and safely through terrain and water.
\subsubsection*{Disguise}
The ability to impersonate someone else or play a role.
\subsubsection*{Evaluate}
Describes the ability to ascertain the value of an object and its true characteristics, mundane, magic, spiritual, etc.
\subsubsection*{Healing}
Use this to undo harm to a character. To heal a character, make an attack roll against the foe that caused the harm's difficulty (or 3 if not specified) each hit recovers a hit. Rolling under the difficulty with all the dice causes 1 hit of damage to the character being healed. Harm from different sources must be treated separately. Healing a character takes a unit of \textbf{time}.
\subsubsection*{Languages}
Represents knowledge in the many tongues of the world. Use this for comprehending and speaking foreign, liturgic or ancient languages.
\subsubsection*{Manuality}
Use this to pick pockets, sleight of hand, perform in stage magic, and make attacks using skill and grace.
\subsubsection*{Mechanisms and Devices}
This skill measures the ability to interact with locks, traps, and the like.
\subsubsection*{Poison}
A skill that abstracts the making and using of poison. When creating a poison, make an attack roll against difficulty 3, the number of hits is equal to the difficulty of the foe the poison is effective for. Poison creation takes a unit of \textbf{time} and a unit of \textbf{loot}. Only make an attack using poison if you are actively using it in a fight (e.g. through a poisoned weapon).
\subsubsection*{Rhetoric}
Use this to express yourself in spoken and nonverbal language. To present arguments, and to strike with words in situations where its possible.
\subsubsection*{Script}
Use this to express yourself in written language, includes the making of both legitimate and forged documents. If you are creative, this can be used as an attack method.
\subsubsection*{Spell}
The ability to perform magic spells, can also serve as an attack method when using an attack spell. Chapter \ref{sec:magic} explains the usage of this skill in detail.
\subsubsection*{Strength}
Use this to lift heavy objects, bend bars, break doors and the like. This skill is also a reliable attack method.
\subsubsection*{Survival}
Use this for foraging while travelling in wilderness.
\subsubsection*{Weapon (S)}
Use this to fight with a weapon. When picking this skill, the character \textbf{must} specify which weapon they are good at. Specifying gives the character +1 to standard d12 rolls using this skill, but not to attack rolls.
\subsubsection*{Other skills}
There are as many skills as there are abilities that might be relevant to characters in a game setting. Players should feel free to discuss adding them to characters if they make sense in the shared fiction. When applicable, specific skills add +1 to standard d12 rolls, but not to attack rolls.
\end{multicols}

\subsection{Ending a Session}
\begin{itemize}
    \item {Fifteen minutes before the session's scheduled end, things get FREAKY;}
    \item {Five minutes before the end of session, the standard play loop stops. Players answer these questions as a group, marking 1 EXPERIENCE for each `Yes':}
    \begin{itemize}
        \item {Did we overcome a difficult or powerful foe?}
        \item {Did we loot, find or acquire something cool?}
        \item {Did we learn something new and important about the world?}
    \end{itemize}
    \item {Each character with six or more EXPERIENCE may level up:}
    \begin{itemize}
        \item {The character spends 6 EXPERIENCE. If their level is 3 or above, they must spend 12 EXPERIENCE to level up;}
        \item {If they spent the appropriate amount of EXPERIENCE, their level increases by 1;}
        \item {They roll on their talent table, acquiring a new ability;}
        \item {The maximum level a character can be is 6;}
    \end{itemize}
    \item {Players roll to see if they got better at skills. For every skill the character used, the PLAYER rolls a d6. If the result is over their current skill value, they permanently increase their skill value by 1.}
    \item {The session then ends. The group may then excitedly schedule the next.}
\end{itemize}
