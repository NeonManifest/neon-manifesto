\newpage
\subsection{Fighter}
The combat specialist, the questing knight, the old soldier. The fighter is the class for those who wish to excel in direct confrontation.

\begin{multicols}{2}
\begin{itemize}
  \item \textbf{HIT VALUE}: 4
  \item \textbf{HITS}: 6
  \item \textbf{MAX BENNIES}: 3
\end{itemize}

\subsubsection*{Appearence}
Roll 1d6 or choose from the table below to determine your Fighter's appearance:

\begin{adjustbox}{width=0.45\columnwidth,center}
\begin{tabular}{|c|l|}
\hline
\textbf{d6} & \textbf{Appearance} \\
\hline
1 & Battle scars \\
2 & Shining armor \\
3 & Tons of tattoos \\
4 & Ripped body \\
5 & Lean and elegant \\
6 & Small and angry \\
\hline
\end{tabular}
\end{adjustbox}

\subsubsection*{Starting Gear}
Fighters always start with their weapon of choice (you choose what the weapon is) and some adventuring gear:

\begin{adjustbox}{width=\columnwidth,center}
\begin{tabular}{|c|l|}
\hline
\textbf{d6} & \textbf{Adventuring Gear} \\
\hline
1 & Rope, whetstone, glowing war paint \\
2 & Grappling hook, flint and steel, singing shield \\
3 & Waterskin, bandages, armor that changes color with mood \\
4 & Torch, rations, boots that never get muddy \\
5 & Bedroll, compass, helmet that translates bird speech \\
6 & Map, spyglass, weapon oil that smells like your enemy's fear \\
\hline
\end{tabular}
\end{adjustbox}

\subsubsection*{Starting Skills}
A Fighter always begins play with a skill value of 2 on their chosen weapon (meaning they attack with 4d6, 2 for the skill, 1 due to the weapon being a tool, 1 for the skill being specific). Roll 1d6 on the table to determine your Fighter's starting skills. You may also choose a line.
\begin{adjustbox}{width=\columnwidth,center}
\begin{tabular}{|c|l|}
\hline
\textbf{d6} & \textbf{Skills} \\
\hline
1 & Strength 2, Climbing, Swimming and Running 1 \\
2 & Manuality 2, Acrobatics 1 \\
3 & Art/Craft (choose) 2, Manuality 1 \\
4 & Awareness 2, Survival 1 \\
5 & Healing 2, Poison 1 \\
6 & Rhetoric 2, Languages 1 \\
\hline
\end{tabular}
\end{adjustbox}

\subsubsection*{Fighter's Deed}
Whenever a Fighter makes an attack, they roll an additional d6. If the Fighter's attack scores a hit and the additional d6's value is above the target's HIT VALUE, the Fighter may perform a deed, which is a feat of great combat prowess. A deed can be anything cool and fighterly your CONDUCTOR agrees to, some sample deeds are:
\begin{itemize}
    \item Tripping or disarming the target;
    \item Pushing the target away;
    \item Gaining +1 to HIT VALUE for the round;
    \item Hurting a specific body part;
    \item Grappling or pinning the target;
    \item Foregoing the deed and gaining a BENNY.
\end{itemize}

\end{multicols}

\begin{table}[!b]
\centering
\small
\setlength{\extrarowheight}{-1pt}
\newcolumntype{Y}{>{\raggedright\arraybackslash}X}
\begin{tabularx}{\textwidth}{|c|Y|}
\hline
\textbf{2d6 Roll} & \textbf{Talent} \\
\hline
2 & \textbf{Iron Constitution:} Increase your maximum HITS by 1.\\
\hline
3-5 & \textbf{Relentless Assault:} When attacking, spend a BENNY to reroll up to X dice showing 1, where X is times you have this talent. \\
\hline
6 & \textbf{Defensive Expertise:} When an attack against you has an exploding d6, spend a BENNY to negate that explosion (initial hit still applies). Use X times, where X is times you have this talent. \\
\hline
7 & \textbf{Combat Focus:} Gain a COMBAT BENNY (+2, attack rolls only). Recover it as you would regular BENNY. Max X COMBAT BENNIES, where X is times you have this talent. \\
\hline
8 & \textbf{Multiattack:} You may attack a second time, with 1d6 and a deed die. Taking this talent multiple times increases the number of successive attacks possible.\\
\hline
9-11 & \textbf{Counter-Strike:} Roll +1d6 on return attacks (after a FOE misses with all attack dice). \\
\hline
12 & Choose any other talent from this table. \\
\hline
\end{tabularx}
\end{table}