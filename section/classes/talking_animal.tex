\newpage
\subsection{Talking animal}
Singing skylarks, cowardly lions, pusses in boots. Talking animals are whimsical creatures of storybooks that occasionaly join adventuring parties.

\begin{multicols}{2}
    \begin{itemize}
      \item \textbf{HIT VALUE}: 3
      \item \textbf{HITS}: 3
      \item \textbf{MAX BENNIES}: 3
    \end{itemize}
    
    \subsubsection*{Species}
    A talking animal can be of any real animal species (finally, a TTRPG that lets you play as an amphioxus). You can also play a larger or tinier version of an animal. Talking aquatic animals are magically adapted to life on land. A talking animal's size determines its statistics:
    \begin{itemize}
        \item {\textbf{Small (dog-sized or smaller)}: HIT VALUE +2;}
        \item {\textbf{Medium (between dog and horse)}: MAX BENNIES +2;}
        \item {\textbf{Large (horse-sized or larger)}: HITS +2;}
    \end{itemize}

    \subsubsection*{Appearence}
    Roll 1d12 or choose from the table below to determine your talking animal's appearance:
    
    \begin{adjustbox}{width=0.55\columnwidth,center}
    \begin{tabular}{|c|l|}
    \hline
    \textbf{d12} & \textbf{Appearance} \\
    \hline
    1 & Glowing fur \\
    2 & Metallic scales \\
    3 & Mangy \\
    4 & Excessively chunky \\
    5 & Wooden prosthetics \\
    6 & Rainbow Feathers \\
    7 & Silly little hat \\
    8 & Neon colors \\
    9 & Wears clothes \\
    10 & Additional eye(s) \\
    11 & (Additional) Horns \\
    12 & Additional tail(s) \\
    \hline
    \end{tabular}
    \end{adjustbox}
    
    \subsubsection*{Gear Restriction}
    Talking Animals start with no gear. Non-medium Talking Animals cannot benefit from tools when making rolls. Small Talking Animals only have 3 item slots.
 
    \subsubsection*{Starting Skills}
    A talking animal's starting skills are determined by their size.
    
    \begin{adjustbox}{width=\columnwidth,center}
    \begin{tabular}{|c|l|}
    \hline
    \textbf{Size} & \textbf{Skills} \\
    \hline
    Small & Acrobatics 2, Awareness 2, Survival 2 \\
    Medium & Strength 2, Awareness 2, Survival 2 \\
    Large & Strength 4, Climbing, Swimming and Running 2\\
    \hline
    \end{tabular}
    \end{adjustbox}

    Also, choose or create another skill relevant to the animal species itself (e.g. poison for a snake, disguise for a fox), your Talking Animal gains 2 on the chosen skill.
\end{multicols}

\begin{table}[!h]
    \centering
    \small
    \setlength{\extrarowheight}{-1pt}
    \newcolumntype{Y}{>{\raggedright\arraybackslash}X}
    \begin{tabularx}{\textwidth}{|c|Y|}
    \hline
    \textbf{2d6 Roll} & \textbf{Talent} \\
    \hline
    2 & \textbf{Fable's Favor}: Once per session, narrate a short fable or moral lesson. If the CONDUCTOR finds it fitting, gain Xd6 temporary BENNIES that can go over your maximum, where X is the number of times you have this talent. \\
    \hline
    3-5 & \textbf{Shapeshifting}: Choose another animal species, and you may turn into it at any time, gaining the benefits and gear restrictions of the other species (your skills are unchanged). You may reroll this talent if you don't like it.\\
    \hline
    6 & \textbf{Strength of the Moon}: Increase your MAX BENNIES by 1 \\
    \hline
    7 & \textbf{Spoken Boon}: Choose another class and roll a talent from its table \\
    \hline
    8 & \textbf{Strength of the Wild}: Increase your maximum HITS by 1 \\
    \hline
    9-11 & \textbf{Storybook Magic}: Choose a fairy tale or fable. Once per session, you can perform a magical feat inspired by that story (e.g., spinning straw into gold, putting others to sleep with a song). The CONDUCTOR determines the effect's potency. You may reroll this talent if you don't like it. \\
    \hline
    12 & Choose any other talent from this table. \\
    \hline
    \end{tabularx}
\end{table}